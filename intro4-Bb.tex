\begin{firstonly}

  This is the last of the introductory labs, and is designed to
  introduce you to the main virtual learning environment used across
  the University, Blackboard, and to give you a chance to practise the
  command line skills that you've learned already, plus a few more, so
  that you're ready for when the regular scheduled labs start. As
  always, don't rush through the material, and if you get at all stuck
  please ask a member of lab staff for help. All the work today should be done using Linux on a desktop PC.

\section{Getting started with Blackboard}
\label{sec:introduction-blackboard}

In the first half of this session you'll be introduced to the
Blackboard Virtual Learning Environment (VLE). Blackboard is designed
to support teaching by providing an online place to upload resources
for course units. It also provides useful tools such as discussion
boards and quizzes. All course units have a Blackboard course unit
site, which provide a first place to look for unit resources; we will
be looking in particular at ways in which it is used in the First Year
Team Project, \courseunit{COMP10120}.


Blackboard has several features which make it well suited to supporting \courseunit{COMP10120}. These include:
\begin{itemize}
\item It's one useful way of communicating with your tutor and members of your tutor group.
\item It allows us to give each group a wiki which you'll use to collaboratively document your group's meetings and the decisions you make as part of  your project work.
\item It provides a structure to help organise activities you should be doing on a week-by-week basis.
\end{itemize}

As a brief introduction to get you started using Blackboard, please take a look at \\ {\small\url{http://studentnet.cs.manchester.ac.uk/ugt/COMP10120/files/BlackboardIntro.pdf}}

\subsection{The COMP10120 course unit site}
\label{sec:comp10120-course-uni}

Now login to Blackboard and navigate to the \courseunit{COMP10120}
site and start to look at how the site is structured and what tools
have been provided. Here are some things you should note about the structure of the site:


\begin{itemize}

\item One of the most important items on the main page is the coloured
table of all your activities for this course unit. The different
phases are colour-coded to help you spot the one you want. The row for
each week contains links to information about what you should be doing
in that week and tools/resources applicable to the week. (Don't forget
to read the \emph{phase overview} first.) The \emph{Other} column
includes resources for your personal use, that you are expected to
complete during this academic year. In particular you should note the
\emph{reflective journals} for each week of the first semester. You
are expected to reflect on the questions detailed inside the journal
each week.

\item General instructions for the course unit are located in the
\emph{General information} area on the front page. Take a look
at each of the links in this area, in particular the \emph{Course
  Mechanics} page where we explain how this course unit is run. 

\item Back on the course unit site, in the left hand panel you will
find \emph{Course Tools}. This link leads you to a number of tools,
including the \emph{Course Discussion Board}. There is also a direct
link to this board on the front page. You should subscribe to this
discussion board now.

\item Also on the left hand side, under \emph{My Groups} on the left
hand side, you should find a link labelled with your tutorial group
name. This is where the resources for your group can be found, including
a discussion board which can be used to communicate with
members of your group and a wiki in which you can document your
work. Again you should subscribe to this discussion board.

\end{itemize}

\subsection{Lab deliverables}
\label{sec:lab-deliverables}

By the end of this section of the lab session, ensure you have completed the following tasks.

\begin{itemize}

\item Posted a welcoming message to your tutor group (see Using
Discussion Boards below).

\item Create your set of practice wiki pages (see Using wikis below).

\end{itemize}

\subsection{Using discussion boards}
\label{sec:using-forums}

Most of you will have used discussion boards, or forums, in some way
or another, whether on your favourite social networking site or on
some other website.

Visit your Group Discussion Board on the \courseunit{COMP10120} site
and click on \emph{Create Thread} to begin a new discussion
thread. Write a short posting to introduce you to your other group
members and your tutor. Let them know where you are from, tell them a
little bit about yourself.

\subsection{Using wikis}
\label{sec:using-wikis}

You have almost certainly come across wikis before, and will no doubt
have looked things up in \wikipedia{Wikipedia}{Wikipedia}, the world's
biggest wiki, many times. Unlike many other online collaboration
systems which constrain users in various ways by pre-determining the
type of content that can be created (for example sites like Instagram
and Flickr are designed for sharing photographs, where as Soundcloud
is for sharing audio), and also categorising users as having different
types of access (e.g. administrators, moderators, regular users and so
forth), the technology behind wikis typically takes a very liberal
approach to both users and content. They generally allow any user to
make any kind of change to any kind of content, and rely on
\textit{social} conventions established by the community to keep
things sane. In particular, every edit, deletion and addition to the
wiki is stored, providing a complete history of wiki changes. Old
versions of pages can be retrieved and compared to new versions of
pages.

As well as creating sites like Wikipedia, wikis are frequently used by
software development teams to document their project. The ability to
look back to previous versions of documents and for multiple authors
to collaborate on producing the documentation without having to worry
too much about `process' is particularly useful in such environments.

During \courseunit{COMP10120} you'll be required to document many
aspects of your work, and you can use a Blackboard group wiki to do that. You
can find this wiki under \emph{My Groups} on the left hand side.

For the purposes of this mini-tutorial you are asked to create some
interlinked wiki pages about yourself. It is important that you follow
this through to the end as it will ensure you know how to do all the
basic tasks needed to help build your own group wiki during the
project.

Start by creating a page in your group wiki by clicking on
\emph{Create Wiki Page}; it probably makes sense to include your name
or initials in its title to avoid confusion with pages created by
other group members. You will be presented with the Blackboard editor
which won't have any content yet.

Select the \emph{Submit} button. You should now see the first page of
your wiki with the content you just added. You can add more content by
selecting the \emph{Edit Wiki Content} tab.

Create a short bullet list containing the items \emph{My hobbies},
\emph{My music} and \emph{My files}.  Save the page again and check
the results, then select the Edit tab again. Now turn the three items
into wiki links. To do this, use the \emph{Create Wiki Page} to make
suitably named new pages (again including something to make them
personal to you, such as your name or initials). Then, back on the
page with the bullet list, select each item of the list in turn and
use strange looking icon with several pieces of paper to link the text
to each of your new pages.

This is the basic process by which you build up a wiki into a series
of linked pages.

You can find lots of information about Blackboard wikis at
{\small\url{https://en-us.help.blackboard.com/Learn/Student/Interact/Wikis}}

Now go back and select the question mark link for the \emph{My music}
link and add some content to this page too. You could write about
music you love (or music you hate!).

In the last part of this mini-tutorial, you will need to add a small
picture to your wiki and a small file. First we need some files to
play with. If you have a small picture that you took yourself, great,
you own the copyright on it, otherwise look for a copyright free image
on the web.

Finally, browse through the remaining documentation on Wikis at the
link given above so that you have an overview of what other
information is there and can refer back to it in the future if
needed. If you have any questions about how to use the wiki tool, post
a message to the Course Discussion Board.


\subsection{Writing your reflective journal}
\label{sec:writing-your-journal}

One of the aims of this course unit is to encourage you to develop the
habit of thinking about the way you are learning and working, and
trying to identify ways in which these could be improved. The process
of \emph{reflection} is key to this, but is not something that comes
naturally to many of us. In most weeks, the course unit site
has a link to an instance of the Journal activity. Any entries you put
into your journal will not be visible to other students. They will be
visible to course unit tutors and course unit organisers, but they
will respect your privacy. 

At some time towards the end of this week, start your journal entry
for the current week. Some brief notes about the process of Reflection
can be found on the above mentioned \emph{Course Mechanics} page,
under \emph{Reflections}. 

That's all we want to say about Blackboard. The second part of this
lab is about using the command line in Linux.
  
\end{firstonly}

